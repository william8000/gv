\documentclass{beamer}
\usepackage[latin1]{inputenc}
\usetheme{Warsaw}
\title{Introduction to subverion}
\author{Markus Steinborn}
\date{November 7th, 2010}

\begin{document}

\begin{frame}
\titlepage
\end{frame}


\begin{frame}{Why subversion}
   \begin{itemize}
      \item All commited versions of the document are available at any time
      \item Several peaple can contribute
      \item Documents are stored at a single place
      \item Subversion provides path based authorization
   \end{itemize}
\end{frame}

\begin{frame}{Basics}
   \begin{itemize}
      \item Changes to the project (as a whole, not per file) are committed under a new version number
      \item For every version, date and time of the commit and the committer is preserved.
      \item If different users change the same document, their change can be merge automatically or manually.
   \end{itemize}
\end{frame}


\begin{frame}{Checkoing out a Working Copy}

A working copy can be checkout out of the repository with\medskip

   \$ svn checkout --username=user --password=pass URL

\end{frame}



\begin{frame}{Updating}
  To update the files of your working copy to the latest versions -- possibly commited by other persons:\medskip
  
  \$ svn update\bigskip
  
  Updating single files is also possible:\medskip
  
  \$ svn update file1 file2 ...\bigskip
  
  To get old versions use the -r flag:\medskip
  
  \$ svn update -r version\_number\medskip
  
  This can be combined with naming files.
\end{frame}

\begin{frame}{Adding and Removing files}

You can add or remove files with:

\begin{itemize}
   \item \$ svn add filename
   \item \$ svn delete filename
\end{itemize}

Note that adding resp. removing the file is done when committing the changes to the repository.
\end{frame}


\begin{frame}{Committing changes}
   After adding, removing and editing files, these changes are committed with:
   \medskip
   
   \$ svn commit --message='meaningful description of your change'
   \bigskip
   
   If you ommit the --message parameter, subversion will start an editor where
   you can input the message.
\end{frame}


\begin{frame}{Examing history}
   \$ svn log\medskip
   
   will show you the commit messages,\bigskip
   
   \$ svn diff -c versionNo
   
   will show you what has been changed in the given version.
\end{frame}


\begin{frame}{And where is the complete manual?}
    \url{http://svnbook.red-bean.com/ }
\end{frame}

\end{document}
